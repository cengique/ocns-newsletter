\section*{International Neuroinformatics Co-ordinating Facility (INCF)}%
\sectionauthor{A collaborative network for open, FAIR, and citable neuroscience}

\begin{figure}[!h]
  \centering
  \includegraphics[width=0.8\textwidth]{images/INCF}
\end{figure}

The mission of INCF network is to promote the uptake of FAIR data management practices in neuroscience through the development of standards and best practices that support open, FAIR, and citable neuroscience.
Specifically, the network aims to:

\begin{itemize}
  \item Provide coordination of global neuroscience infrastructure through the development and endorsement of standards and best practices in support of open and FAIR (Findable, Accessible, Interoperable, Reusable) neuroscience
  \item Support neuroscience as discipline to move towards FORCE (FAIR, Open, Research-object based, and Citable Ecosystem) through the development of community resources and the provision of training opportunities
  \item Encourage neuroscience as discipline to move towards FORCE
  \item Promote the advancement and continued development of neuroinformatics
\end{itemize}

INCF provides a community environment which has developed over the past decade with the engagement of neuroscience, neuroinformatics, and data science researchers, tool developers, and infrastructure developers from academic groups across the globe.
This environment has proven conducive to initiating standardization efforts between not just the large brain projects but within the global neuroscience community as a whole.

\section*{Connect with INCF}
\sectionauthor{\vspace{-4ex}}
\vspace{1ex}
\begin{center}
  \href{https://twitter.com/INCFOrg}{\vcenteredinclude{./images/X-logo}}~
  \href{https://www.facebook.com/INCForg/}{\vcenteredinclude{./images/Facebook_icon_2013.svg}}~
  \href{https://bsky.app/profile/incforg.bsky.social}{\vcenteredinclude{./images/Bluesky_Logo}}~
  \textbf{\Large @INCFOrg}
  \hspace{1cm}\href{https://neuromatch.social/@INCF}{\vcenteredinclude{./images/mastodon}}~
  \textbf{\Large @INCF}
  \hspace{1.2cm}\href{https://www.linkedin.com/company/international-neuroinformatics-coordinating-facility-incf-}{\vcenteredinclude{./images/LinkendIn}}~
  \textbf{\Large @INCF}
\end{center}
\vspace{1ex}

\begin{center}
\begin{tcolorbox}[colback=lighterOrange, colframe=black, width=0.7\textwidth, boxrule=0.1mm]
\begin{center}
  \href{https://mailchi.mp/incf/signup}{Subscribe to the INCF newsletter}\\\vspace{1ex}\href{https://www.incf.org/join-incf}{Become a member}
\end{center}
\end{tcolorbox}
\end{center}

\section*{Why should you support INCF?}
\sectionauthor{\vspace{-4ex}}
There is a very real need for coordination of global neuroscience data, which is satisfied by the activities of the INCF network.
Standardizing global neuroscience can be done in a cost-effective manner but it cannot be done without support.
Support for infrastructures such as INCF is crucial, and granting agencies must allow and encourage grantees to participate in activities such as the INCF network in order to develop and implement data management and data sharing workflows.
The responsibility lies not only on funders: we, the neuroscience community, have as much responsibility for collecting and curating data as we do to ensure data can be effectively shared.
Participating in the INCF network is an opportunity to build the capacity that will enable neuroscience teams to take on this data sharing responsibility.


\textbf{OCNS is a proud institutional member of the INCF.}
