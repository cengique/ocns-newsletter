% break into separate file, since we expect this to be along section
\section*{OCNS: Member Updates: Community Development}%
\sectionauthor{Updates related to community development from OCNS members}

\begin{itemize}
    \item \textbf{The Capo Caccia Worskhops toward Neuromorphic Intelligence}

        Submitted by: Mihai A Petrovici

        \url{https://capocaccia.cc/en/event/ccnw24/landing-page/}

        The goal of the CCNW workshops is to promote the neuromorphic approach to designing technologies, establish an international community, and to encourage collaboration amongst small groups, in order to achieve the kind of technical advances which could only otherwise happen in well-funded industrial labs.

        The CCNW has an open format, whose intention is to encourage creativity and exploration of ideas and projects in a relaxed and intellectually open environment. Although there is a skeleton program that sets a default route through the two weeks, ad hoc deviations from or elaborations of this basic program are encouraged. Discussion groups and projects arise dynamically. There are no formal lectures. Instead, the morning consist of two 1.5 hr discussion sessions in which a few discussants will make short contributions to the topics in order to ignite more general interaction. Although the sessions of the skeleton program have assigned moderators and discussants, these persons should also be seen as defaults. Whiteboards and overhead tablets are available for drawings. Formal presentations with prepared media (such as Powerpoint slides) are strictly forbidden. The daily program includes a late afternoon sports break, and happy hour.


    \item \textbf{The Lu.i educational neurons}

        Submitted by: Mihai A Petrovici

        \url{https://physiologie.unibe.ch/~petrovici/group/lui.aspx}

        With an increasing presence of science throughout all parts of society, there is a rising expectation for researchers to effectively communicate their work and, equally, for teachers to discuss contemporary findings in their classrooms. While the community can resort to an established set of teaching aids for the fundamental concepts of most natural sciences, there is a need for similarly illustrative experiments and demonstrators in neuroscience. We therefore introduce Lu.i: a parametrizable electronic implementation of the leaky-integrate-and-fire neuron model in an engaging form factor. These palm-sized neurons can be used to visualize and experience the dynamics of individual cells and small spiking neural networks. When stimulated with real or simulated sensory input, Lu.i demonstrates brain-inspired information processing in the hands of a student. As such, it is actively used at workshops, in classrooms, and for science communication. As a versatile tool for teaching and outreach, Lu.i nurtures the comprehension of neuroscience research and neuromorphic engineering among future generations of scientists and in the general public.

    \item \textbf{Neuroscience Gateway (NSG)}

        Submitted by: Amitava Majumdar

        \url{https://www.nsgportal.org}

        NSG project provides free and open access to supercomputing resources. NSG enables modeling, simulation and data processing (e.g. EEG, MEG, fMRI etc.) research in neuroscience by lowering the administrative and technical barriers that currently make it difficult for investigators to use large scale computing resources. It provides access to popular neuroscience tools, pipelines, data processing software and libraries.


\end{itemize}
\newpage
\begin{itemize}
    \item \textbf{Advanced Scientific Programming in Python Summer School}

        Submitted by: Athanasia \enquote{Nassi} Papoutsi

        \url{https://aspp.school}

        The Institute of Molecular Biology and Biotechnology of the Foundation for Research and Technology Hellas (IMBB-FORTH) at Heraklion, Crete, Greece hosted the 16th Advanced Scientific Programming in Python Summer School from August 26th to August 31st, 2024. The Summer School, kindly funded by Tubingen AI Center, was attended by 30 participants from all over the world. The interactive lectures, pair programming sessions, and coding tournaments made for an unforgettable experience. A huge thanks to our amazing participants, faculty, and organizers for bringing so much energy and enthusiasm!

\end{itemize}

